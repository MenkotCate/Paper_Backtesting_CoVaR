\documentclass[12pt]{article}
\usepackage{amsmath,amssymb,multirow, float, amsthm}
\usepackage{graphicx,psfrag,epsf}
\usepackage{enumerate}
\usepackage{natbib}
\usepackage{url} % not crucial - just used below for the URL

%\pdfminorversion=4
% NOTE: To produce blinded version, replace "0" with "1" below.
\newcommand{\blind}{0}

% DON'T change margins - should be 1 inch all around.
\addtolength{\oddsidemargin}{-.5in}%
\addtolength{\evensidemargin}{-.5in}%
\addtolength{\textwidth}{1in}%
\addtolength{\textheight}{1.3in}%
\addtolength{\topmargin}{-.8in}%
\newtheorem{theorem}{Theorem}

\begin{document}

%\bibliographystyle{natbib}

\def\spacingset#1{\renewcommand{\baselinestretch}%
{#1}\small\normalsize} \spacingset{1}


%%%%%%%%%%%%%%%%%%%%%%%%%%%%%%%%%%%%%%%%%%%%%%%%%%%%%%%%%%%%%%%%%%%%%%%%%%%%%%

\if0\blind
{
  \title{\bf Backtesting CoVaR}
  \author{Author 1\\
    Department of YYY, University of XXX\\
    and \\
    Author 2 \\
    Department of ZZZ, University of WWW}
  \maketitle
} \fi

\if1\blind
{
  \bigskip
  \bigskip
  \bigskip
  \begin{center}
    {\LARGE\bf Backtesting (Delta-)CoVaR}
\end{center}
  \medskip
} \fi

\bigskip
\begin{abstract}
%In this paper we propose a new backtesting method for Conditional Value at Risk ($CoVaR$), a systemic risk measure introduced by \citet{adrian}. Instead of testing the subset of $CoVaR$ predictions in which the condition is satisfied (the approach used in previous research), we predict the $CoVaR$ as a function of its condition, and evaluate the predicted $CoVaR$ at the realisation of this condition. This allows us to perform tests on the full set of $CoVaR$ predictions. In our Monte Carlo study we show that our test is useful in a realistic setting and that it outperforms previous $CoVaR$ backtesting methods. Finally, we apply our test to real world data and use it to compare the performance of different $CoVaR$ prediction methods.
We propose a new backtesting method for Delta-CoVaR. This measure of systemic risk is defined as the increase in Value of Risk of a financial institution when the financial system is in distress (i.e., when it violates its own VaR) compared to a non-distress state (when its return is at the median). By relying on a conditional version of the probability integral transform, we are able to exploit the entire sample of CoVaR predictions, rather than just the periods in which the system is in distress. A Monte Carlo study demonstrates that the new test has dramatically increased power vis-a-vis existing tests. An empirical application demonstrates the use of the test in a real-world scenario.
\end{abstract}
{\it Keywords:}  Backtest; CoVaR; Systemic risk.
\vfill

\newpage
\spacingset{1.45} % DON'T change the spacing!




%%% 1 INTRODUCTION %%%

\section{Introduction}
\section{Setup and Notation}
\section{A new Backtest for $\Delta$CoVaR}
\section{Finite Sample Properties}
\subsection{Monte Carlo Design}
\subsection{Size}
\subsubsection{Power}
\subsubsection{Misspecified Contemporaneous Dependence}
\subsubsection{Misspecified Dynamics}
\subsubsection{Misspecified Tails}
\section{Application}
\subsection{Data}
\subsection{Estimation Methods}
\subsection{Results}
\section{Conclusions}
\bibliographystyle{agsm}

\bibliography{Bibliography-MM-MC}
\end{document}

