\documentclass[12pt]{article}
\usepackage{amsmath,amssymb,multirow, float, amsthm}
\usepackage{graphicx,psfrag,epsf}
\usepackage{enumerate}
\usepackage{natbib}
\usepackage{url} % not crucial - just used below for the URL

%\pdfminorversion=4
% NOTE: To produce blinded version, replace "0" with "1" below.
\newcommand{\blind}{0}

% DON'T change margins - should be 1 inch all around.
\addtolength{\oddsidemargin}{-.5in}%
\addtolength{\evensidemargin}{-.5in}%
\addtolength{\textwidth}{1in}%
\addtolength{\textheight}{1.3in}%
\addtolength{\topmargin}{-.8in}%
\newtheorem{theorem}{Theorem}

\begin{document}

%\bibliographystyle{natbib}

\def\spacingset#1{\renewcommand{\baselinestretch}%
{#1}\small\normalsize} \spacingset{1}


%%%%%%%%%%%%%%%%%%%%%%%%%%%%%%%%%%%%%%%%%%%%%%%%%%%%%%%%%%%%%%%%%%%%%%%%%%%%%%

\if0\blind
{
  \title{\bf Backtesting CoVaR}
  \author{Author 1\\
    Department of YYY, University of XXX\\
    and \\
    Author 2 \\
    Department of ZZZ, University of WWW}
  \maketitle
} \fi

\if1\blind
{
  \bigskip
  \bigskip
  \bigskip
  \begin{center}
    {\LARGE\bf Backtesting CoVaR}
\end{center}
  \medskip
} \fi

\bigskip
\begin{abstract}
\noindent In this paper we propose a new backtesting method for Conditional Value at Risk ($CoVaR$), a systemic risk measure introduced by \citet{adrian}. Instead of testing the subset of $CoVaR$ predictions in which the condition is satisfied (the approach used in previous research), we predict the $CoVaR$ as a function of its condition, and evaluate the predicted $CoVaR$ at the realisation of this condition. This allows us to perform tests on the full set of $CoVaR$ predictions. In our Monte Carlo study we show that our test is useful in a realistic setting and that it outperforms previous $CoVaR$ backtesting methods. Finally, we apply our test to real world data and use it to compare the performance of different $CoVaR$ prediction methods.\end{abstract}

\noindent%
{\it Keywords:}  Model Validation Testing Conditional VaR
\vfill

\newpage
\spacingset{1.45} % DON'T change the spacing!




%%% 1 INTRODUCTION %%%

\section{Introduction}


Since the financial crisis, there has been a lot of attention for systemic risk, the risk that one institution in financial distress can contaminate the entire financial system. Federal Reserve Governor Daniel Tarullo formulated systemic importance in a 2009 testimony before the Senate Banking, Housing, and Urban Affairs Committee as follows:\footnote{Source: https://www.federalreserve.gov/newsevents/testimony/tarullo20090723a.htm}

\textit{``Financial institutions are systemically important if the failure of the firm to meet its obligations to creditors and customers would have significant adverse consequences for the financial system and the broader economy."}

The most widely used risk measure in finance was (and still is) the Value at Risk ($VaR$), which is defined as the maximum return loss of a financial institution within a certain probability. The $VaR$ focuses on one financial institution in isolation and does not capture its risk contribution to the financial system. Therefore, there has been a lot of interest in risk measures which take the interaction between financial institutions into account. Systemic risk measures aim to capture the amount of risk in the financial system that can be attributed to a financial institution.

Several systemic risk measures have been developed. \citet{billio} use the time variation in the number of principal components necessary to explain a certain fraction of the volatility of the return of the financial system, to determine the time variation in the level of interconnectedness between individual financial institutions. They analyse the directionality of shocks with Granger causality tests. 

\citet{huang} construct a systemic risk measure from a hypothetical insurance premium against distress of the financial system based on the probability of default of individual institutions and their asset correlations. 

\citet{acharya1} define the Systemic Expected Shortfall ($SES$) as the Expected Shortfall of an institution given that the financial system is in crisis. (The Expected Shortfall is defined as the expected loss of an institution given that its loss is above it's $VaR$ level.)  \citet{acharya2} extend this definition to the expected amount of capital that a firm needs in case of a financial crisis.

This paper focuses on a systemic risk measure developed by \citet{adrian}. It is called the \textit{Difference in Conditional Value at Risk} or $\Delta CoVaR$. The $\Delta CoVaR$ of an institution $j$ is defined as the $VaR$ of the financial system conditional on $j$ being in distress minus the $VaR$ of the financial system conditional on the return of $j$ being at its median. This definition captures the systemic risk in line with the definition mentioned earlier, the amount of systemic risk that can be attributed to institution $j$.

A lot of further research on $\Delta CoVaR$ has been published after the first working paper of Adrian \& Brunnermeier was published in 2008. \citet{girardi}, \citet{copulas}, and \citet{mesbacktest} provide backtesting methods to test whether an estimated $(\Delta) CoVaR$ series is specified correctly. 

The general idea of these methods is to use conventional $VaR$ backtesting methods on the subset of $CoVaR$ predictions in which the condition is satisfied (i.e.  where institution $j$ is in distress). A disadvantage of these methods is that testing only this subset (often only about $1\%$ or $5\%$ of the predictions) could lead to limited test power in realistic sample sizes. 

In this paper, we propose a new $CoVaR$ backtesting method. Instead of using a subset in which the condition is satisfied, we predict the $CoVaR$ as a function of its condition (the return of $j$) and evaluate the predicted $CoVaR$ at the realisation of this condition (the observed return of $j$). This allows us to perform tests on the full set of $CoVaR$ predictions.

The remainder of this paper is organised as follows. Section 2 formally defines the $\Delta CoVaR$ and discusses different estimation methods. Section 3 introduces a previous $CoVaR$ backtesting method as well as our new $CoVaR$ backtesting method. 

In Section 4, we use Monte Carlo simulation to assess the finite sample properties of our test. We find that our test is useful in a realistic setting and that it has better small sample properties than previous tests. In Section 5 we apply our test to real world data and compare the performance of three $CoVaR$ estimation methods. Finally, Section 6 makes some concluding remarks and some suggestions for further research.


  
\section{Conclusion}


In this paper a new method for backtesting $Quantile$ $CoVaR$ predictions is introduced. Instead of testing the subset of $Tail$ $CoVaR$ predictions where the condition is satisfied (the approach used in previous research), we predict the $Q$-$CoVaR$ as a function of its condition (the return of $j$) and evaluate the predicted $Q$-$CoVaR$ at the realisation of this condition (the observed return of $j$). This allows us to perform tests on the full set of $Q$-$CoVaR$ predictions. 

By definition, the observed return of $j$ should have no additional explanatory power on the hit series of a correctly specified $Q$-$CoVaR$ evaluated at the at the observed return of $j$. We include this property in the tests, which allows us to test whether there is misspecification in the dependency structure.

The Monte Carlo simulation shows that our test is useful in a realistic setting when we want to test whether a $CoVaR$ prediction method is specified correctly. It can detect both a distributional and a dynamic misspecification. Our test also outperforms previous $CoVaR$ backtesting methods in our simulation. The test is a little over-sized which is partly caused by the parameter estimation error. In line with \citet{nonlineartest}, we find that the linear test and the logistic test have a similar performance.

We predicted $Q$-$CoVaR$ series for the five largest US banks, where we used the Dow Jones US Financials Index as a proxy for the financial system.  We applied our tests to compare the performance of the quantile regression, the Asymmetric GARCH-DCC model with Gaussian innovations, and the Asymmetric GARCH-DCC model with Student-$t$ innovations.

The results indicate that the Asymmetric GARCH-DCC model with Student-$t$ innovations has the best ability to describe the underlying dynamics and that the quantile regression has the worst performance (out of the three estimation methods). A plausible explanation is that in the quantile regression, the dependency coefficient is fixed over time, while the Asymmetric GARCH-DCC model allows for time-varying correlation. 

We also observe that the Student-$t$ innovations are outperforming the Gaussian innovations in particular in the end of the tail (i.e. the 99\% $CoVaR$). This may be due to the fatter tails of the Student-$t$ distribution. 

Even if one prefers to use $T$-$CoVaR$ (based on an inequality condition) for a certain application (e.g. because of its monotonicity in the dependence parameter), our test might still be useful. One approach could be to first predict the $Q$-$CoVaR$ using different models and to apply our tests to find which model has the best performance. Then, using the preferred model, the $T$-$CoVaR$ can be predicted. This does not necessarily result in the optimal $T$-$CoVaR$ predictions. However, it may give better finite sample results than testing only a small subset of the $T$-$CoVaR$ predictions directly.

\citet{bayesian} apply Bayesian updating to the quantile regression coefficients to enable them to describe the time-varying dependency. An interesting approach for future research would be to use their prediction methods and to apply our tests to the Bayesian $Q$-$CoVaR$ predictions.

In this paper we focussed on the conditioning direction ``$VaR$ of the financial system given the return loss of $j$" ($CoVaR^{system|j}$). However, our test is equally applicable to the opposite definition where the $CoVaR$ is defined as the $VaR$ of an institution given that the financial system is in distress ($CoVaR^{i|system}$). This is also called ``\textit{Stressed Value at Risk}". Banks are required to calculate a $Stressed$ $VaR$ by the Basel II regulations. The $Stressed$ $VaR$ should be based on 10-day ahead predictions of the 99\% quantile \citep[][p. 14]{baselii}. Therefore, it would be an interesting approach for future research to investigate the performance of our test for 10-day ahead predictions.





\bibliographystyle{agsm}

\bibliography{Bibliography-MM-MC}
\end{document}
